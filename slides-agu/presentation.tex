\documentclass[
  10pt,     % set default generic font size
  %handout   % ignores all the \pause commands
]{beamer}
\usetheme[block=fill]{metropolis}

% -----------------------------------------------------------------------------
% Title and Author Information -- Need to Edit
% -----------------------------------------------------------------------------

\title{H52F-05H: Robust Adaptation to\\Multi-Scale Climate Variability}
\subtitle{Toward Better Water Planning and Management in an Uncertain World I}
\date{19 October 2018}
\author{\alert{James Doss-Gollin}$^1$, David J. Farnham$^2$, Scott Steinschneider$^3$, Upmanu Lall$^1$}
\institute{
  $^1$Columbia University Department of Earth and Environmental Engineering\\
  $^2$Carnegie Institution for Science\\
  $^3$Department of Biological and Environmental Engineering, Cornell University}
\titlegraphic{\hfill\includegraphics[height=1.25cm]{SeasCrown_blue.png}}

% -----------------------------------------------------------------------------
% Package Configuration -- Don't Necessarily Need to Edit
% -----------------------------------------------------------------------------

% Packages with Options
\usepackage[english]{babel}

% Package List
\usepackage{
  array,                              % for custom table widths
  appendixnumberbeamer,               % don't count appendix slides in progress bar
	booktabs,                           % for better (alternative?) tables
  natbib,                             % references!
	physics,                            % for better notation
  siunitx,                            % for SI notation
}

% cool fonts
\usepackage{fontspec}
\usepackage{fontawesome5}

% figures
\usepackage{graphicx}
\graphicspath{{../fig/}} % can add more

% Fixed-width columns
\usepackage{array}
\newcolumntype{L}[1]{>{\raggedright\let\newline\\\arraybackslash\hspace{0pt}}m{#1}}

% Change the captions
\setbeamerfont{caption}{size=\scriptsize}

% macros
\usepackage{xspace}
\newcommand*{\eg}{e.g.\@\xspace}
\newcommand*{\ie}{i.e.\@\xspace}
\makeatletter
\newcommand*{\etc}{%
    \@ifnextchar{.}%
        {etc}%
        {etc.\@\xspace}%
}
\makeatother
\newcommand{\usd}[1]{\SI[round-precision=2,round-mode=places,round-integer-to-decimal]{#1}[\$]{}}
\newcommand{\normal}{\mathcal{N}}

% Biblatex Setup using a file called library.bib
\setbeamertemplate{bibliography item}[text] % don't print the symbols

% this has to come last
\usepackage{cleveref}

% -----------------------------------------------------------------------------
% BEGIN DOCUMENT HERE
% -----------------------------------------------------------------------------

\begin{document}

% TITLE PAGE
\maketitle

\begin{frame}{Thanks to...}
  \begin{itemize}
    \item Co-authors
    \begin{itemize}
      \item Upmanu Lall
      \item David J. Farnham
      \item Scott Steinschneider
    \end{itemize}
    \item Funders
    \begin{itemize}
      \item NSF GRPF
      \item Columbia University Fu Foundation SEAS
      \item DoD SERDP
    \end{itemize}
    \item Conveners
  \end{itemize}
\end{frame}

\begin{frame}{NYC After Sandy \citep{CityofNewYork:2013uh}}
  \begin{columns}%[T]
    \begin{column}{0.5\textwidth}
      \begin{figure}
        \includegraphics[width=\columnwidth]{three-barriers.png}
      \end{figure}
      \begin{itemize}
        \item Environmental, social impacts
        \item Permanent and inflexible: $\mathcal{O} \qty(\$25 \text{billion})$ is ``locked-in''
        \item Potential for over- or under-design
      \end{itemize}
    \end{column}
    \pause
    \begin{column}{0.5\textwidth}
      \begin{figure}
        \includegraphics[width=\columnwidth]{flood-concept.png}
      \end{figure}
      \begin{itemize}
        \item Local resilience and vulnerability reduction
        \item Portfolio of long- and short-term investments
        \item Update portfolio regularly
      \end{itemize}
    \end{column}
  \end{columns}
\end{frame}

\begin{frame}{Planning Periods}
  Projects are designed for finite ``planning periods'' $M$ over which climate risk must be estimated.
  \begin{columns}
    \begin{column}{0.6\textwidth}
      \begin{itemize}
        \item Success of ``mega-project'' depends on climate conditions $M \geq \SI{50}{years}$ in the future
        \item Success of flexible, non-structural strategy (\ie financial instrument) depends on climate conditions $M \leq \SI{5}{years}$ in the future
      \end{itemize}
    \end{column}
    \begin{column}{0.4\textwidth}
      \begin{figure}
        \includegraphics[width=\columnwidth]{Bias-Variance-Sketch.pdf}
        \caption{Consequences of over- or under-estimating climate risk and associated uncertainty}
      \end{figure}
    \end{column}
  \end{columns}
\end{frame}


\begin{frame}{Drivers of Future Climate Risk}
  The physical drivers of hazard depend on the projection horizon ($M$).
  However, our ability to identify these drivers depends on information available (\eg, the length of an $N$-year observational record).
  \begin{figure}
    \centering
    \includegraphics[width=\textwidth,height=0.55\textheight,keepaspectratio=true]{conceptual-sketch.pdf}
    \caption{Schematic description of (a) intrinsic and (b) estimation uncertainty.}
  \end{figure}
\end{frame}

\begin{frame}{Real-World Climate Signals}
  Real-world hydroclimate systems vary on many time and space scales
  \begin{figure}
    \centering
    \includegraphics[width=\textwidth]{observed-lfv.pdf}
    \caption{
      (a) \SI{500}{year} reconstruction of summer rainfall over Arizona from LBDA \citep{Cook:2010bz}.
      (b) A \SI{100}{year} record of annual-maximum streamflows for the American River at Folsom.
      (c),(d): wavelet global (average) spectra.
    }\label{fig:observed-lfv}
  \end{figure}
\end{frame}

\begin{frame}{Research Objective}
  How well can one identify and predict risk associated with cyclical and secular climate signals for the design life ($M$ years) and the probability of over- or under-design of a climate adaptation strategy given limited information (for which an $N$-year record is a proxy).
  In particular, we consider:
  \begin{itemize}
    \item the specific roles of $M$ and $N$;
    \item the choice of estimation strategy; and
    \item the form of the underlying climate signal.
  \end{itemize}
\end{frame}

\section{Stylized Experiments}

\begin{frame}{Research Plan}
  \begin{figure}
    \centering
    \includegraphics[width=\textwidth]{flowchart.pdf}
    \caption{Flow chart of methods used.}
  \end{figure}
  \pause
  \emph{N.B.}: models used for generating sequences and estimating risk chosen for \emph{interpretability}, not general \emph{validity} in applications.
  Results are similar with other models.
\end{frame}

\begin{frame}{Example Sequences}
  \begin{figure}
    \centering
    \includegraphics[width=\textwidth,height=0.65\textheight,keepaspectratio=true]{Example-NINO3-M100-N50.pdf}
    \caption{
      Synthetic flood sequences (blue) and estimated 50\%, 95\% intervals (gray).
      Three scenarios are used for generating sequences (columns) and three models are used for estimating risk (rows).
      Here $N=50$ and $M=100$.
    }
  \end{figure}
\end{frame}

\begin{frame}{Stationarity: LFV Only}
  Estimating a trend when none exists may lead to substantial over-estimation of risk, particularly for short $N$ or long $M$.
  Explicitly modeling LFV improves estimates.
  \begin{figure}
    \centering
    \includegraphics[width=\textwidth,height=0.65\textheight,keepaspectratio=true]{lfv-only-nino3-bias-variance.pdf}
    \caption{
      Expected estimation bias and variance for sequences with LFV but zero trend.
    }\label{fig:lfv-nino3-bias-variance}
  \end{figure}
\end{frame}

\begin{frame}{Nonstationarity: Secular Change Only}
  Stationary models under-estimate risk, particularly for long $M$.
  For short $M$ and short $N$, the uncertainty by estimating a trend is not worth the bias reduction.
  \begin{figure}
    \centering
    \includegraphics[width=\textwidth,height=0.65\textheight,keepaspectratio=true]{secular-only-nino3-bias-variance.pdf}
    \caption{
      Expected estimation bias and variance for sequences with secular change only (no LFV).
    }\label{fig:secular-nino3-bias-variance}
  \end{figure}
\end{frame}

\begin{frame}{Nonstationarity: Secular Change + LFV}
  Long $M$ requires trend estimation, but without sufficient data (long $N$) trend estimates are strongly biased.

  \begin{figure}
    \centering
    \includegraphics[width=\textwidth,height=0.65\textheight,keepaspectratio=true]{lfv-secular-nino3-bias-variance.pdf}
    \caption{
      As \cref{fig:secular-nino3-bias-variance} but for sequences generated with both LFV and secular change.
    }\label{fig:lfv-secular-nino3-bias-variance}
  \end{figure}
\end{frame}

\section{Discussion}

\begin{frame}{Idealized Experiments $\iff$ Real World}
  The idealizations and ``toy'' models used here are analogues for more complex real-world situations.
  \begin{table}
    \begin{tabular}{L{0.425\textwidth}L{0.525\textwidth}}
      \toprule
      Analysis & Real World \\\midrule
      $N$-year record & Total informational uncertainty of an estimate \\\midrule
      Statistical models of increasing complexity and \# parameters & Statistical and dynamical model chains of increasing complexity and \# parameters \\\midrule
      LFV comes from ENSO only & LFV comes from many signals \\\midrule
      Linear trends & Secular changes of unknown form \\\midrule
      Trends and LFV Independent & Trends and LFV modes interact\\
      \bottomrule
    \end{tabular}
  \end{table}
\end{frame}

\begin{frame}{Key Findings}
  Depending on the specific climate mechanisms that impact a particular site, and the predictability thereof, the cost and risk associated with a sequence of short-term adaptation projects may be lower than with building a single, permanent structure to prepare for a worst-case scenario far into the future:
  \begin{itemize}
      \item As uncertainty increases, adaptation strategies with short planning periods are preferred
      \item Risk profile of short-$M$ projects dominated by LFV \citep{Jain:2001hz, Hodgkins:2017hw}
      \item When the planning period is long, trends must be estimated explicitly -- requiring more data
  \end{itemize}
\end{frame}

\begin{frame}{A Shameless Pitch!}
  I'm working to quantify these principles.
  Specifically:
  \begin{itemize}
    \item Statistical models for integrating paleo records, observations, and GCMs towards climate risk projection
    \item Financial valuation of the ``option'' to build a particular project
    \item Optimizing portfolios of long- and short-term investments
  \end{itemize}
  If you have a working model of your system and are making long-/short-term investment choices I'd \alert{love to talk} further!
\end{frame}

% -----------------------------------------------------------------------------
% QUESTIONS, BIBLIOGRAPHY
% -----------------------------------------------------------------------------

\begin{frame}[allowframebreaks]{References}
  \renewcommand*{\bibfont}{\footnotesize}
  \renewcommand{\bibsection}{}
  \nocite{DossGollin:TjTkb07T}
	\bibliographystyle{agufull08}
  \bibliography{library}
\end{frame}

\begin{frame}[standout]
  \alert{Thanks for your attention!}\\\vspace{1.5cm}
  \begin{tabular}{rl}
    \faIcon[regular]{twitter},\faIcon[regular]{github} & \url{@jdossgollin} \\
    \faIcon[regular]{envelope} & \url{james.doss-gollin@columbia.edu}\\
    \faIcon[regular]{paperclip} & \url{www.jamesdossgollin.me}
  \end{tabular}
\end{frame}

% -----------------------------------------------------------------------------
% BACKUP SLIDES
% -----------------------------------------------------------------------------

\appendix
\renewcommand{\thefigure}{A\arabic{figure}}
\setcounter{figure}{0}
\renewcommand{\theequation}{A\arabic{equation}}
\setcounter{equation}{0}
\renewcommand{\thetable}{A\arabic{table}}
\setcounter{table}{0}

\section{Generating Synthetic Streamflow Sequences}

\begin{frame}{Equations for Synthetic Streamflow Generation}
  First
  \begin{equation} \label{eq:lognormal}
    \log Q(t) \sim \normal \qty(\mu(t), \sigma(t)).
  \end{equation}
  Where $\sigma(t) = \xi \mu(t)$, with $\sigma(t) \geq \sigma_\text{min} > 0$.
  Then,
  \begin{equation}\label{eq:nino3}
    \mu(t) = \mu_0 + \beta x(t) + \gamma \qty(t - t_0),
  \end{equation}
  and where $x(t)$ is NINO3.4 index from realistic ENSO model \citep{Zebiak:1987cl,Ramesh:2016hf}
\end{frame}

\begin{frame}{Spectrum of LFV Used}
  \begin{figure}
    \includegraphics[width=\textwidth,height=0.6\textheight,keepaspectratio=true]{enso_wavelet}
    \caption{Wavelet spectrum of (sub-set of) ENSO model used to embed synthetic streamflow sequences with low-frequency variability}
  \end{figure}
\end{frame}

\section{Climate Risk Estimation}

\begin{frame}{Stationary LN2 Model}
  Treat the $N$ historical observations as IID draws from stationary distribution
  \begin{align}\label{eq:ln2-stationary}
    \begin{split}
      \log Q_\text{hist} & \sim \normal \qty(\mu, \ \sigma) \\
      \mu &\sim \normal \qty(7, 1.5) \\
      \sigma &\sim \normal^+ \qty(1, 1)
    \end{split}
  \end{align}
  where $\normal$ denotes the normal distribution and $\normal^+$ denotes a half-normal distribution.
  Fit in Bayesian framework using stan \citep{Carpenter:2017ke}.
\end{frame}

\begin{frame}{Trend LN2 Model}
  Treat the $N$ historical observations as IID draws from log-normal distribution with linear trend
  \begin{align}\label{eq:ln2-trend}
    \begin{split}
      \mu &= \mu_0 + \beta_\mu \qty(t - t_0) \\
    \log Q_\text{hist} & \sim \normal \qty(\mu, \ \xi \mu) \\
    \mu_0 & \sim \normal \qty(7, 1.5) \\
    \beta_\mu & \sim \normal \qty(0, 0.1) \\
    \log \xi & \sim \normal \qty(0.1, 0.1)
    \end{split}
  \end{align}
  where $\xi$ is an estimated coefficient of variation.
  Also fit in stan.
\end{frame}

\begin{frame}{Hidden Markov Model}
  Two-state HMM \citep[see][]{Rabiner:1986jk} implemented using pomegranate python package \citep{Schreiber:2017tg}.
  See package documentation for reference.
\end{frame}

\end{document}
