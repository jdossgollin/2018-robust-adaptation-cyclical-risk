\begin{block}{Key Findings}
    \begin{itemize}
        \item If risk uncertainty increases into future (i.e., \cref{fig:conceptual-sketch}) then adaptation strategies with short $M$ are preferred \cite{Haasnoot:2013im, Zeff:2016hx}
        \item Risk profile of short-$M$ projects dominated by \acrlong{lfv} \cite{Jain:2001hz, Hodgkins:2017hw}
        \item Even though the \gls{hmm} is an imperfect analog for \gls{enso}, it performs well for short $M$
        \item When $M$ is long, trends must be estimated explicitly (but this requires more information -- longer $N$)
        \item Depending on the specific climate mechanisms that impact a particular site, and the predictability thereof, the cost and risk associated with a sequence of short-term adaptation projects may be lower than with building a single, permanent structure to prepare for a worst-case scenario far into the future
    \end{itemize}
\end{block}
