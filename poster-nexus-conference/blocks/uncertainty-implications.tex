\begin{block}{Estimation Bias and Variance}
    The insurance premium for a financial risk mitigation instrument on event $X$ can be parameterized as
    \begin{equation*}
        P = \mathbb{E}\qty[X] + \lambda \sigma \qty[X]
    \end{equation*}
    Thus, if an estimate has a positive bias and overestimates uncertainty, the instrument may be too expensive for the user.
    Conversely, if an estimate has negative bias and underestimates uncertainty, it will be likely to fail (\cref{fig:conceptual-bias-variance}).
    A key question is thus \emph{whether the limited information in an $N$-year observational record permits the identification and projection of cyclical climate variability and secular change, and what the resulting bias and uncertainty portends for risk mitigation instruments with a planning period ranging from a few years to several decades.}
    \begin{figure}
        \centering
        \includegraphics[width=0.6\textwidth]{Bias-Variance-Sketch.pdf}
        \caption{
        Consequences of model bias or incorrect model representation of uncertainty.
        }\label{fig:conceptual-bias-variance}
    \end{figure}
\end{block}