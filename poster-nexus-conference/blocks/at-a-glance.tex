\begin{block}{At a Glance}
    Assessing the utility of a particular risk mitigation instrument involves projecting the climate hazard against which the instrument protects over the $M$-year planning period.
    \begin{itemize}
        \item Climate risk varies in time over a project’s finite planning period, $M$, which may be 1-5 years for a financial risk mitigation instrument and 30-100 years for a structural instrument.
        \item Cyclical climate variability (anthropogenic climate change) dominates near-term (long-term) climate risk
        \item Successful climate adaptation requires prediction of short- and long-term climate variability over the finite planning period $M$
    \end{itemize}
    An often neglected point is that the sources of predictability differ between projects with long and short planning periods.
    \emph{We present a set of stylized experiments to assess how well one can identify and predict risk associated with cyclical and secular climate signals for the design life ($M$ years) and the probability of over- or under-design of a climate adaptation strategy based on these projections.}
\end{block}
