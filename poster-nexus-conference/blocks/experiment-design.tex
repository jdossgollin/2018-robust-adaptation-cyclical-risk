\begin{block}{Experiment Design}
    \begin{figure}
        \centering
        \includegraphics[width=\textwidth]{flowchart.pdf}
        \caption{
          Framework for assessing flood risk from synthetic streamflow sequences.
        }\label{fig:methods-summary}
      \end{figure}
      Stylized experiment design (\cref{fig:methods-summary}):
      \begin{enumerate}
        \item For each, generate many synthetic streamflow sequences embedded with one of
        \begin{enumerate}
          \item Secular trend only
          \item \gls{lfv} from \gls{enso} only
          \item both \gls{lfv} and secular trend
        \end{enumerate}
        plus stochastic variability.
        \item Fit using simple Bayesian model
        \item Evaluate estimation bias and variance across all synthetic streamflow sequences
        \item Repeat for many combinations of $N$ (length of historical record: proxy for informational uncertainty) and $M$ (project planning period)
      \end{enumerate}
      This approach is illustrated in \cref{fig:example-fit}
      \begin{figure}
        \includegraphics[width=\textwidth]{Example-NINO3-M100-N50.pdf}
        \caption{
          An illustration of the estimation procedure.
          A single streamflow sequence with $N=50$ and $M=100$ is shown for each of the three cases (secular only, \gls{lfv} only, and secular plus \gls{lfv}) considered.
          The blue line shows the observed sequence.
          The gray shading indicates the 50\% and 95\% confidence intervals using each of the three fitting methods discussed.
          The horizontal black line indicates the flood threshold.
        }\label{fig:example-fit}
      \end{figure}
\end{block}