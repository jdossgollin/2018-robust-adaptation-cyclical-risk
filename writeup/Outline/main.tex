\documentclass[11pt]{article}

\usepackage{xspace}
\newcommand{\eg}{e.g.\@\xspace}
\newcommand{\ie}{i.e.\@\xspace}
\newcommand{\normal}{\mathcal{N}}
\newcommand{\GEV}{\operatorname{GEV}}
\newcommand{\iid}{i.i.d.\@\xspace}
\makeatletter
\newcommand*{\etc}{%
    \@ifnextchar{.}%
        {etc}%
        {etc.\@\xspace}%
}
\makeatother

\newcommand{\expectation}{\ensuremath{\mathbb{E}}}
\newcommand{\variance}{\ensuremath{\mathbb{V}}}
\newcommand{\probability}{\ensuremath{\mathbb{P}}}
\newcommand{\invlogit}{\ensuremath{\text{logit}^{-1}}}
% Packages With Options
\usepackage[letterpaper, margin=1in]{geometry} % page
\usepackage[english]{babel} % language
\usepackage[utf8]{inputenc} % input
\usepackage[inline]{enumitem} 
\usepackage[section]{placeins} % floats appear in the section they were defined in

% Other Package calls
\usepackage{
  array, % for fixed-width tables
  amssymb, amsmath, % for math
  authblk, % for authors
	booktabs, % for nice tables
  csquotes, % better biblatex
  graphicx, % for figures
  indentfirst, % indent the first line
  physics, % for physics notation
  ragged2e, % for better alignment of text
  siunitx, % for SI notation
  nth, % for 1st, 2nd, 3rd, nth
}

% PACKAGE SETTINGS
\setlength{\RaggedRightParindent}{\parindent} % fix ragged right
\graphicspath{{fig/}{../../fig/}}
\sisetup{round-mode=figures,round-precision=3,scientific-notation=false}
\allowdisplaybreaks{} % let the align environment span multiple pages

% arrays with fixed width
\newcolumntype{L}[1]{>{\raggedright\let\newline \\ \arraybackslash\hspace{0pt}}m{#1}}

\title{Climate Change Adaptation: Uncertainty Associated with Sequential Finite Period Decisions under Nonstationarity}
\author[1,2]{James Doss-Gollin}
\author[1,2]{Upmanu Lall}
\author[1,2]{David Farnham}
\affil[1]{Columbia Water Center, Columbia University}
\affil[2]{Department of Earth and Environmental Engineering, Columbia University}

\usepackage[hidelinks]{hyperref}
\usepackage{cleveref}
\usepackage[
  backend=biber,
  doi=true,
  url=false,
  isbn=false,
  style=authoryear-comp,
  %style=nature,
  natbib=true,
  backref=false,
  maxbibnames=3,
  maxcitenames=2,
  uniquename=false,
  uniquelist=false,
  sorting=ynt
]{biblatex}
\renewcommand{\refname}{References}
\renewbibmacro{in:}{}
\AtEveryBibitem{\clearfield{month}\clearfield{pages}}

\addbibresource{../library.bib}

% -----------------------------------------------------------------------------
% AUTHOR AND TITLE AND TITLEPAGE
% -----------------------------------------------------------------------------

\begin{document}
\maketitle
%\RaggedRight

\begin{abstract}
  Hydroclimatic systems exhibit organized low-frequency and regime-like variability at multiple time scales, causing the risk associated with climate extremes such as floods and droughts to vary at multiple timescales.
  Despite broad recognition of this nonstationarity, there has been little theoretical development of ideas for the design and operation of infrastructure considering the regime structure of such changes and their potential predictability.
  This theory is well-suited to analysis of flood adaptation strategies with fixed project operation period $M$, but not for analysis of sequential decision strategies with $M$.
  In this paper we illustrate some basic considerations of uncertainty and risk analysis associated with the sequential decisions that may need to be made for climate risk adaptation in a nonstationary adaptation.
  We simulate synthetic data and fit the resulting simulations in a Bayesian context using stationary and non-stationary models to show that \texttt{RESULT}.
  This is a first conceptual step to the development of a full sequential decision analysis approach under nonstationarity.
\end{abstract}


\section{Introduction}

Hydroclimatic systems vary on many timescales, including regime-like variability and secular variability  \citep{Milly2008,Merz2014,Hurst1951,Sveinsson2005,Hodgkins2017,Matalas2012}.
This multi-scale variability greatly complicates the estimation of future flood sequences, leading to ``low confidence in projections of changes in fluvial floods'' \citep{IPCC2012}.
Yet the same memory and low-frequency variability that complicate projecting flood sequences far into the future may also impart some predictability at short timescales \citep{Jain2001}.
The existence of this information gap suggests that approaches for adapting to hydroclimate risks, such as floods, which have short project operation periods, may be preferential in some cases to large and permanent projects with long project operations.

Classical methods for design of infrastructure first specify a return period $T$ (\ie 100 years), and then design a project that protects against this $T$-year event.
Much of the flood frequency analysis (FFA) literature has tended to focus on the details of flood frequency estimation and on trend detection associated with climate change and land use cover \citep{Merz2008,Kidson2016,Merz2014}.
The mathematical problem of interest then becomes estimation of the probability distribution $f(p_T)$, where $p_T$ is binary event that the threshold $Q_T$ is exceeded:
\begin{equation*}
  p_T(t) \equiv \probability \qty[ Q(t) \geq Q_T] \qqtext{for} t = \qty[t_0 + 1, \ldots, t_0 + M]
\end{equation*}
and where $t_0$ is the first year forecast and $M$ is the project operation period.
Although in a stationary setting an unbiased estimator of $Pr \qty[ Q(t) \geq Q_T]$ converges asymtotically to $1 - \frac{1}{T}$, the asymmetric uncertainty distribution of the estimate has also prompted the consideration of the potential for over- or under-design for floods if the conditional mean of the estimated distribution of $Q_T$ is used for design \citep{Stedinger1997}\footnote{Vogel198x citation}.

Alternatively, risk-based design methods \citep[RBDM; see][]{Rosner2014} recognize that the appropriate level of design  may itself be a decision variable.
\citet{Lall1987} considered the uncertainty in the frequency with which floods may exceed the design level for different values of $p$ (equivalent to different values of $T$), and also for different sample sizes for estimation $N$ and different project operation periods $M$.
The duration $M$ may reflect the physical life of a proposed infrastructure element, or its economic life.
In the context of financial instruments which may be used for risk mitigation, $M$ specifies the length of the contract and $Q_T$ the threshold at which a payout is triggered.
In such a setting, the mathematical problem is to estimate the future distribution of $Q(t)$ rather than just the future distribution of the binary outcome $p_T$.

Under classical assumptions of stationarity, all estimates are assumed independent of time.
Yet even under stationarity, \citet{Jain2001} showed that low-frequency variability can lead to a high probability of surprise -- particularly when the length of the record used for estimation $N$ is short.
This finding is particularly important to keep in mind given that low-frequency change has been found to be large as compared to detected trends in the observed record \citep{Hodgkins2017}.

Of course, the hydroclimate systems are not stationary, due to climate, river modification, and land use change \citep{Milly2008,Merz2014}.
However, predicting future hydrologic behavior in a particular location is inherently difficult given limited data and complex, highly nonlinear physics.
Physical scaling arguments for an intensified hydrologic cycle under climate change \citep[see][]{Muller2011,OGorman2015} do not in general hold over land \citep{Byrne2015,Shaw2016}.
Attempts to use a ``model chain'' approach emcompassing:
\begin{enumerate*}[label=(\roman*)]
  \item emission scenario;
  \item general circulation model (GCM);
  \item downscaling;
  \item hydrological catchment model; and
  \item flood frequency analysis,
\end{enumerate*}
typically with bias correction applied at several steps, can lead to results which are difficult to interpret in a probabilistic context \citep{Dankers2009,Ott2013,Merz2014,Dittes2017}.
In particular, the downscaling and bias correction methods typically applied assume stationary relationships (\ie between GCM rainfall and observed rainfall) which have no physical basis, particularly in a changing climate.
Recent work has also sought to shorten the above model chain by modeling $Q(X(t))$, where $X$ is a set of climate state variables from a GCM run, and where the relationship between $X$ and $Q$ can more reasonably be assumed stationary \citep{Hall2014,Delgado2014,Silva2016}.
Alternatively, purely statistical approaches have extended classical flood frequency analysis by incorporating a time trend in the parameters of statistical distributions \citep{Obeysekera2014,Vogel2011,Serinaldi2015,Strupczewski2001}.
In this case, however, the analyst can choose not only the distribution but also  the time parameterization, if any, for each parameter, leading to many researcher degrees of freedom \citep[``forking paths'' or ``multiple comparisons'';][]{Gelman2013} and exacerbating the problems caused by lack of theory for model choice even under nonstationarity \citep{Kidson2016}.

The wide variety of methods, each with its own advantages and disadvantages, for estimating the future distribution of flooding $p \qty(Q(t))$ illustrates the theoretical need for approaches to incorporate uncertain estimates of future floods into a decision framework.
As we have argued, however, low-frequency variability or system memory may lead to less uncertain probabilistic forecasts at short times than long times.
There is thus \emph{a critical theoretical need for decision frameworks which can compare projects of different operation period} using an arbitrary choice of model(s) of future flood behavior.
Sequential decision models \citep[see][]{Russell2003,Howard1960} allow decision-makers to optimize not only what action to take, but also when to take it.
Sequential decision models have been used in the hydrological literature for sea wall optimization \citep{Lickley2014}\footnote{need some citations for sequential decisions}.

In this context, we consider financial risk mitigation instruments in which the price is a function of the amount of coverage $C$, the expected probability that the insurance is triggered $\expectation \qty(p_T) \equiv \frac{1}{T}$, and the variance (or more generally full uncertainty distribution) associated with this estimate $\variance \qty(p_T)$.
Understanding the full uncertainty associated with thresholds of interest is relevant to many other climate adaptation problems, including infrastructure design for flood management, because we are faced with the task of estimating the infrastructure's probability of failure $p_T$ for the next $M$ years, as well as the uncertainty and bias in this estimate.
However, to properly analyse it we need detailed information on complex cost and loss structures; we thus focus on the illustrative example of financial instruemnts for clarity.

In the following sections we illustrate some very basic considerations of uncertainty and risk analysis associated with the sequential decisions that may need to be made for climate risk adaptation in a nonstationary environment.
This is a first conceptual step to the development of a full sequential decision analysis approach under nonstationarity.
We proceed as follows $\ldots$.

\section{Methods} \label{sec:methods}

We use a highly idealized model to generate sequences of annual-maximum floods using a known generating mechanism, then fit these sequences in a Bayesian framework to three flood models (stationary, nonstationary, and hidden markov model) to evalaluate their performance under different scenarios (defined by observational record length $N$ and project planning period $M$) and criteria.

\subsection{Synthetic Flood Sequence Generation}


Extensive discussion in the literature has considered the appropriate choice of distribution for annual-maximum flood sequences.
Although three-paremeter models are better able to represent extreme values than two parameter models \citep{IACWD1982,Vogel1996}, we assume that flood sequences are \emph{conditionally} lognormal, given a mean and standard deviation which vary in time:
\begin{equation} \label{eq:lognormal}
  \log Q(t) \big| \mu(t),\sigma(t) \sim \normal \qty(\mu(t), \sigma(t)).
\end{equation}
where $\normal$ denotes a normal distribution and $\mu(t), \sigma(t)$ are the location and scale parameters, respectively.

In large river basins, extreme river flooding typically requires the large-scale transport and convergence of moisture,\footnote{Is this worth adding in} which often requires specific climate mechanisms.
For example, major river floods in the Ohio River Basin (USA) in the 20th century exhibit remarkably consistent large-scale circulation features in the weeks preceding the flood peak \citep{Nakamura2012,Robertson2015}.
This is consistent with findings that have linked flood risk around the world with well-known climate mechanisms such as El Ni\~{n}o-Southern Oscillation (ENSO) \citep{Ward2014,Emerton2017}.
To capture this influence, some recent studies have incorporated a climate index or set of climate indices $X$ as predictor variables for streamflow \citep{Delgado2014,Silva2016,Sun2014,Griffis2007}\footnote{Farnham et al}.
For a more comprehensive review, we refer the reader to \citet{Hall2014}.\footnote{Farnham et al}

ENSO \citep[see][for a comprehensive review]{Sarachik2010} varies in a quasi-oscillatory manner with variability on the order of \SIrange{4}{6}{year} modulated by low-frequency behavior.
Though the time scales are specific to ENSO, this pattern of quasi-oscillatory behavior modulated by low-frequency behavior is typical of many other climate phenomena.
A \SI{100000}{year} integration of the Cane-Zebiak model \citep{Zebiak1987} was used to produce a monthly NINO3.4 index \footnote{We greatfully acknowledge the contribution of Nandini Ramesh} with stationary forcing.
To create an annual time series, we average the May-July months to create a time series $X(t)$ where $t$ is taken in annual steps $1, \ldots, \num{100000}$.

Though statistical models relating climate indices and streamflow maxima typically require strong assumptions (\ie a stationary relationship between predictor and predictand; see \texttt{Farnham et al}), this framework is the starting point of our synthetic flood generation approach.
We partition $\mu(t)$ into a dynamical component, driven by the variability of the climate predictor $X(t)$, and a trend component, represented by a linear increase in time.
This trend component may represent changes in land use, global temperature change, or river modification \citep[see][]{Merz2014}.
Though this is a highly simplified model neglecting many important mechanisms, it enables us to generate many synthetic sequences of $\qty[M + N]$ years which exhibit chaotic but organized quasi-periodic behavior (regime behavior or low-frequency variability) plus a trend term, thereby capturing many essential features of hydrological variability.
Because the computational cost is small, it is easy to generate sets of sequences following the procedure
\begin{enumerate}
  \item Choose the parameters $M,N,\alpha,\beta,\gamma$
  \item Select the ``zero'' time of analysis $t_0 \in \qty[N, \ldots, \num{100000} - M]$ so that the time used is $t_{\text{use}} = t_0 - N + 1, \ldots, t_0 - M$ (note that this indexing treats $t_0$ as part of the $N$-year observed record)
  \item Subset $x(t) = X(t \in t_{\text{use}})$
  \item Let $\mu(t) = \beta x(t) + \gamma \qty(t - t_0)$
  \item Assume a constant coefficient of variation so that $\sigma(t) = \alpha \mu(t)$
  \item Draw $\log Q(t) \sim \normal \qty(\mu(t), \sigma(t))$
\end{enumerate}

In \cref{sec:sequence-realistic} we show sequences of streamflow generated by this model for some choices of $M,N,\alpha,\beta,\gamma$ and discuss the processes which they can reasonably represent.

\subsection{Distributional Model \label{sec:estimation}}

Because we do not, in general, know the true distribution of annual-maximum floods, we compare a stationary and non-stationary parameterizations of the generalized extreme value (GEV) statistical distribution (eq.~\ref{eq:GEV}) for estimating the distribution of $Q(t)$ for the future $M$-year period.
\begin{equation}
  \label{eq:GEV}
  f(x \big| \mu, \sigma, \xi) = \frac{1}{\sigma}
    \begin{cases}
      \qty[1 + \xi \frac{x - \mu}{\sigma}] ^ {(-\flatfrac{1}{\xi})-1} \exp \qty[-\qty(1+\xi \frac{x - \mu}{\sigma})^{\flatfrac{-1}{\xi}}], & \xi \neq 0 \\
      \exp \qty[-\frac{x-\mu}{\sigma}] \exp \qty[- \exp \qty(- \frac{x-\mu}{\sigma})], & \xi = 0
    \end{cases}
\end{equation}
Where $\mu$ is the location parameter, $\sigma$ is the shape parameter, and $\xi$ is the scale parameter of the GEV distribution.
For each simulation, the first $N$ years $\qty(t=[t_0-N+1, \ldots, t_0])$ are treated as observations.
Then, once the appropriate time parameterization is selected (\cref{tab:model-fitting}), the parameters are fit in a Bayesian framework to fully represent the posterior uncertainty.
Computation is carried out in the \texttt{python} environment, with simulation using the \texttt{numpy} package \citep{vanderWalt2011} and fitting using the No-U-Turn Sampler (NUTS) \citep{Hoffman2014} as implemented in the \texttt{stan} probabilistic programming language \citep{Carpenter2016}.
Weak (uninformative) priors are chosen based on recommendations from \citet{Martins2000}; the exact parameterizations are described in following sections which describe specific experiments.

\subsection{Hidden Markov Model \label{sec:HMM}}

A Hidden Markov Model (HMM) is a statistical Markov model in which the system being modeled is assumed to be a Markov process with unobserved (\ie hidden) states \citep{Rabiner1986}.
The (unobserved) states evolve following a first-order Markov process, and the observed variable (\ie streamflow) depends only on the observed state.
As such, HMMs are a structured prediction method which extend general mixture models to sequences of data, where position in the sequence is relevant.
Because this structure allows the hidden state to represent hydroclimatological ``regimes'' \citep{Reinhold1982,Michelangeli1995,Merz2014}, HMMs have have been used extensively to model hydrological variables such as streamflow \citep{Bracken2016}.
HMMs have also been used to model the evolution of climate states such as ENSO.\footnote{Julian's work}

To supplement the distributional models described in \cref{sec:estimation}, we also fit observed streamflow sequences using a HMM as implemented in the python \texttt{pomegranate} package \citep{Schreiber2016}.
The model is fit to the $N$ observed years of data using the Baum-Welch algorithm, assuming that the data follow a log-normal distribution conditional only on the state.
To choose the number of hidden states, a log probability score is used.
Future floods are estimated by simulating future states from the resulting matrix and then drawing floods log-normally using the estimated parameters.

\begin{table}[b]
  \begin{center}
    \begin{tabular}{L{1.0in} L{3.0in} L{1.5in}}
      \toprule
        Model Name & Description & Relevant Section \\
      \midrule
        stationary & $Q \sim \GEV \qty(\mu, \sigma, \xi)$ & \cref{sec:estimation} \\
        location trend & $Q \sim \GEV \qty(\mu_0 + \beta_\mu \qty(t-t_0), \sigma, \xi)$ & \cref{sec:estimation} \\
        %full trend & $Q \sim \GEV \qty(\mu_0 + \beta_\mu \qty(t-t_0), \sigma_0 + \beta_\sigma \qty(t-t_0), \xi_0 + \beta_\xi \qty(t-t_0))$ \\
        HMM & log-normal Hidden Markov Model & \cref{sec:HMM} \\
      \bottomrule
    \end{tabular}
  \end{center}
  \caption{Summary of models used for fitting \label{tab:model-fitting}}
\end{table}


\subsection{Flood Estimation and Uncertainty}

Finally, we consider a contract in which an insurer agrees to pay the entire coverage cost $C$ every time that the trigger threshold $Q_T$ is exceeded.
Although $Q_T$ is in general a decision variable, we treat it as fixed.
Then, using the posterior distribution of the parameters of the chosen statistical model (from \cref{tab:model-fitting}) we estimate the probability that $Q(t) \geq Q_T$ for $t \in \vb{t}$, which is equivalent to $p_T(t)$.

For a particular year, the fair price for the insurance premium is $\expectation \qty(p_T) C$.
However, there is typically also a risk premium associated with the policy that prices the uncertainty in this estimate (since the trigger $Q_T$ is specified in the contract and thus not uncertain).
This risk premium is proportional to the variance of $p_T$, or more generally to its uncertainty distribution.
The insurance premium for a particular year is thus
\begin{equation} \label{eq:premium}
  Z(t) = C \qty[\expectation \qty(p_T(t)) + R \variance \qty(p_T(t))]
\end{equation}
where $R \geq 0$ is a coefficient describing the cost of tue uncertainty in the estimate of $p_T(t)$.

In this paper we consider the price of a contract of $M$ years to be simply the summed price of the contract for each individual year.
We assume that the discount rate is equal to inflation (\ie the inflation-adjusted discount rate is zero) for simplicity.
A more comprehensive pricing mechanism for the financial instrument could incorporate more or different information about the uncertainty in the estimate, and should consider the price of the $M$ years as a portfolio (\ie should consider correlation between the different years).
Such an approach, however, is not necessary for this conceptual study.

\section{Are Generated Sequences Realistic? \label{sec:sequence-realistic}}

In this section we discuss the sequences of streamflow that are generated and look at their ACF, PACF, distribution, wavelet spectrum, \etc and compare to some that have been identified in the literature.


\section{Stationary, Static Risk}

Consider static risk and a stationary process, and partiuclarly implications of $M$ and $N$.
Much of this is already laid out in \citet{Lall1987}.

\section{Stationary, Static Risk}

Consider static risk and a non-stationary process.
Estimate using drift or nonstationary models using full $N$ of the most recent $N$ (\ie what Vogel is proposing).
Illustrate via simulation for different signal to noise ratio of nonstationary terms and $M$ and $N$.

\section{Dynamic Risk}

Consider dynamic risk, \ie updating and sequential decisions, a nonstationary generative process, and $M$ and $N$ considerations

\section{Discussion}

Hopefully something to discuss?

\section{Summary}

Some interesting conclusions



% -----------------------------------------------------------------------------
% END HERE
% -----------------------------------------------------------------------------
\clearpage
\printbibliography

\end{document}
