\section{Synthetic Flood Frequency Estimation}

\begin{enumerate}
  \item Points to hit in this section:
  \begin{enumerate}
    \item How do different estimation models perform for different $M,N$?
    \item What is the inherent uncertainty (\ie using correct model) in estimation for different $M,N$?
  \end{enumerate}
  \item First experiment: visualizing distributions
  \begin{enumerate}
    \item Points to hit with this experiment:
    \begin{enumerate}
      \item Bad estimation models lead to biased results that don't correctly hit the uncertainty
      \item There is a bias-variance tradeoff associated with making predictions at different $M$ for different lengths of records -- and in a setting with a lot of anthropogenic and low-frequency variability with how long your record $N$ is.
    \end{enumerate}
    \item Procedure for this experiment:
    \begin{enumerate}
      \item For several (two or three) values of $M,N$, create many sequences of $(M+N)$ years from the ``true'' model
      \item For each sequence:
      \begin{enumerate}
        \item Fit the model in a Bayesian framework
        \item For each draw from the posterior, calculate the probability of exceeding some threshold $\tau$ in the future $M$ years. Can consider subtracting out the value estimated from the ``true'' sequence for normalization purposes
        \item Calculate the expected value of exceeding this threshold and some metric of the spread (95\% confidence interval width?) associated with these MCMC draws of $p(y > \tau)$
      \end{enumerate}
      \item Plot these two distributions (over many sequences), separately for the two or three values of $M$ and $N$.
      \item We can repeat for a couple of estimation strategies.
    \end{enumerate}
  \end{enumerate}
  \item Second experiment: focusing on $M,N$
  \begin{enumerate}
    \item Points to hit with this experiment:
    \begin{enumerate}
      \item How does varying $M$ and $N$ affect the results above?
      \item What are implications of different $M$ and $N$ for estimation procedures used?
    \end{enumerate}
    \item Procedure for this experiment:
    \begin{enumerate}
      \item Create a grid of $M$ and $N$. For each grid cell, create many $(M+N)$-year sequences from the ``true'' model. For each sequence:
      \begin{enumerate}
        \item Fit the model in a Bayesian framework
        \item For each draw from the posterior, calculate the probability of exceeding some threshold $\tau$ in the future $M$ years. Can consider subtracting out the value estimated from the ``true'' sequence for normalization purposes
        \item Calculate the expected value of exceeding this threshold
      \end{enumerate}
      \item Calculate the expected value by averaging across all sequences for the given $M,N$
      \item Repeat over all grid cells
      \item Plot the grid
      \item Repeat for several estimation strategies
    \end{enumerate}
  \end{enumerate}
\end{enumerate}

% -----------------------------------------------------------------------------
% Loss Function
% -----------------------------------------------------------------------------

\section{Synthetic Utility Estimation}

\begin{enumerate}
  \item Points to hit in this section
  \begin{enumerate}
    \item Uncertainty decreases with $N$, if you are fitting using a reasonable model
    \item For short $M$ the uncertainty is less in a world with strong memory than in a world with more short-term variance and long-term climate change signal.
    \item For long $M$ the uncertainty in a world with strong long-term climate change signal may be less than in a world with strong persistence.
  \end{enumerate}
  \item Experiment: comparing two projects with different $M$. Procedure:
  \begin{enumerate}
    \item Define a cost function for the proposed decisions. Can be simplified -- assume exposure is constant(-ish?)
    \item Define a few representative values of $N$ (\SIlist{10;30;50;100}{year})
    \item For each decision, and each value of $N$:
    \begin{enumerate}
      \item create many $(M+N)$-year sequences from the ``true'' model. For each sequence:
      \begin{enumerate}
        \item Fit the estimation model in a Bayesian framework
        \item For each draw from the posterior, calculate the expected utility
      \end{enumerate}
      \item Plot the distribution for this decision and value of $N$.
    \end{enumerate}
  \end{enumerate}
  \item Experiment: time-varying flood exposure. Procedure: repeat above analysis except rather than varying $N$ (pick a standard like 30 or 50 years), vary the flood exposure (\ie between 7\% growth per year and 2\% decline per year). Thus show that even with a strong climate change signal, the distribution of future utility is extremely variable for a large (costly) project with long $M$ -- this should address ``our friends who want to build Trump walls for 2100 GCM projections''
\end{enumerate}



% -----------------------------------------------------------------------------
% Some Results
% -----------------------------------------------------------------------------

\section{Preliminary Results}

Here are some results for a few combinations of true models and fits.
For each model:
\begin{enumerate}
  \item For each (of 1000) sample of the parameters from the MCMC simulation and each (of 1000) sequences of streamflow that were synthetically generated, the $T$-year event. Here $T=100$. Plot distribution of $T$-year event.
  \item Look at exceedances in the true data of the estimated $T$-year event
  \item Calculate and plot expected number of exceedances per year
  \item Calculate and plot probability of at least one exceedance
\end{enumerate}

\subsection{Stationary Data fit to Stationary Model}
I set $\mu=10,\sigma=2$
\begin{figure}[ht]
  \centering
  \includegraphics[width=5.5in]{stationary_stationary_x_star_distribution.pdf}
  \caption{Distribution of $100$-year event $X^*$ for stationary model with stationary fit}
\end{figure}
\begin{figure}[ht]
  \centering
  \includegraphics[width=5.5in]{stationary_stationary_expected_exceedances.pdf}
  \caption{Exceedance per year for stationary model with stationary fit}
\end{figure}
\begin{figure}[ht]
  \centering
  \includegraphics[width=5.5in]{stationary_stationary_risk_distribution.pdf}
  \caption{Probability of at least one exceedance for stationary model with stationary fit}
\end{figure}

\subsection{Low-Frequency Variability fit to Stationary Model}
I set $\mu_10$, $\sigma=1$, $A=[1, 0.25]$, $T=[40, 5]$.
\begin{figure}[ht]
  \centering
  \includegraphics[width=5.5in]{lowfreq_stationary_x_star_distribution.pdf}
  \caption{Distribution of $100$-year event $X^*$ for low-frequency model with stationary fit}
\end{figure}
\begin{figure}[ht]
  \centering
  \includegraphics[width=5.5in]{lowfreq_stationary_expected_exceedances.pdf}
  \caption{Exceedance per year for low-frequency model with stationary fit}
\end{figure}
\begin{figure}[ht]
  \centering
  \includegraphics[width=5.5in]{lowfreq_stationary_risk_distribution.pdf}
  \caption{Probability of at least one exceedance for low-frequency model with stationary fit}
\end{figure}

\subsection{Remaining}

\begin{enumerate}
  \item Fitting simulated data:
  \begin{enumerate}
    \item I have working code to generate fit
    \item How to think about $X^*$ ($T$-year event): can we calculate it analytically as function of $\hat{\mu_0},\hat{\beta},\hat{\sigma},M$ or resolve using monte carlo simulation?
    \item Build \texttt{PyMC3} model for low-frequency variability with known $T$ (easy)
    \item Add linear trends (easy)
  \end{enumerate}
  \item Loss function
  \begin{enumerate}
    \item We've considered the problem of insuring a dam -- need to implement
  \end{enumerate}
\end{enumerate}
