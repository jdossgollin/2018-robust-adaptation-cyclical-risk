\usepackage[letterpaper, margin=0.85in]{geometry} % page
\usepackage[english]{babel} % language
\usepackage[utf8]{inputenc} % input
\usepackage[inline]{enumitem}

\usepackage{
  amssymb, amsmath, % for math
  physics, % for physics notation
  siunitx, % for SI notation
  csquotes, % better biblatex
  listings, % for code
  graphicx, % for figures
  mathpazo, % for palatino font
  ragged2e, % for better alignment of text
  indentfirst, % indent the first line
  array, % for fixed-width tables
	booktabs, % for nice tables
  comment, % comment a lot at once
  authblk, % for authors
}

% PACKAGE SETTINGS
\setlength{\RaggedRightParindent}{\parindent} % fix ragged right
\graphicspath{{fig/}{../../fig/}}
\sisetup{round-mode=figures,round-precision=3,scientific-notation=false}
\allowdisplaybreaks % let the align environment span multiple pages

% for arrays
\newcolumntype{L}[1]{>{\raggedright\let\newline\\\arraybackslash\hspace{0pt}}m{#1}}

% Keep figures in sub-sections
\usepackage[section]{placeins} % floats appear in the section they were defined in
\begin{comment}
\makeatletter
\AtBeginDocument{%
  \expandafter\renewcommand\expandafter\subsection\expandafter
    {\expandafter\@fb@secFB\subsection}%
  \newcommand\@fb@secFB{\FloatBarrier
    \gdef\@fb@afterHHook{\@fb@topbarrier \gdef\@fb@afterHHook{}}}%
  \g@addto@macro\@afterheading{\@fb@afterHHook}%
  \gdef\@fb@afterHHook{}%
}
\makeatother
\end{comment}
